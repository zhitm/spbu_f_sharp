\documentclass{article}
\usepackage{graphicx} % Required for inserting images
\usepackage{mathtools}

\title{Lambda homework 3}
\author{Zhitnuhina Maria}
\usepackage[russian]{babel}

\date{\today}

\begin{document}

\maketitle
 
\textbf{Task 1.} 
 $ ((\lambda a.(\lambda b.b b) (\lambda b.b b)) b) ((\lambda c.(c\:b)) (\lambda a.a))\xrightarrow[\beta]{}((\lambda b.b)(\lambda b.b))((\lambda c.(c b))(\lambda a.a))\xrightarrow[\beta]{}((\lambda b.b)(\lambda b.b))((\lambda a.a) b)\xrightarrow[\beta]{}(\lambda b.b)(\lambda b.b) b$

\textit{У полученного терма нет НФ, поскольку $\beta$-редукции не изменяют его. 
Но мы шли по нормальной стратегии, а она обязательно приводит к НФ, если она существует. Следовательно, у исходного терма нет НФ.}

\textbf{Task 2.}
$ S K K = (\lambda x y z.x z (y z)) (\lambda x y.x) (\lambda x y.x) 
= (\lambda x.\lambda y.\lambda z.x z (y z))(\lambda x.\lambda y.x) (\lambda x.\lambda y.x)
\xrightarrow[\beta]{} (\lambda y.\lambda z.(\lambda x.\lambda y.x) z (y z)) (\lambda x.\lambda y.x)
\xrightarrow[\beta]{} \lambda z.(\lambda x.\lambda y.x) z ((\lambda x.\lambda y.x) z)
\xrightarrow[\beta]{} \lambda z.(\lambda x.\lambda y.x) z \lambda y.z
\xrightarrow[\beta]{} \lambda z.(\lambda y.z) \lambda y.z
\xrightarrow[\beta]{} \lambda z.z = I
$

\end{document}
